\iffalse
\let\negmedspace\undefined
\let\negthickspace\undefined
\documentclass[journal,12pt,twocolumn]{IEEEtran}
\usepackage{cite}
\usepackage{amsmath,amssymb,amsfonts,amsthm}
\usepackage{algorithmic}
\usepackage{graphicx}
\usepackage{textcomp}
\usepackage{xcolor}
\usepackage{txfonts}
\usepackage{listings}
\usepackage{enumitem}
\usepackage{mathtools}
\usepackage{gensymb}
\usepackage{comment}
\usepackage[breaklinks=true]{hyperref}
\usepackage{tkz-euclide} 
\usepackage{listings}
\usepackage{gvv}                                        
\def\inputGnumericTable{}                                 
\usepackage[latin1]{inputenc}                                
\usepackage{color}                                            
\usepackage{array}                                            
\usepackage{longtable}                                       
\usepackage{calc}                                             
\usepackage{multirow}                                         
\usepackage{hhline}                                           
\usepackage{ifthen}                                           
\usepackage{lscape}

\newtheorem{theorem}{Theorem}[section]
\newtheorem{problem}{Problem}
\newtheorem{proposition}{Proposition}[section]
\newtheorem{lemma}{Lemma}[section]
\newtheorem{corollary}[theorem]{Corollary}
\newtheorem{example}{Example}[section]
\newtheorem{definition}[problem]{Definition}
\newcommand{\BEQA}{\begin{eqnarray}}
\newcommand{\EEQA}{\end{eqnarray}}
\newcommand{\define}{\stackrel{\triangle}{=}}
\theoremstyle{remark}
\newtheorem{rem}{Remark}
\begin{document}

\bibliographystyle{IEEEtran}
\vspace{3cm}

\title{Probability Assignment}
\author{EE22BTECH11028-Katherapaka Nikhil$^{*}$% <-this % stops a space
}
\maketitle
\newpage
\bigskip
\renewcommand{\thefigure}{\theenumi}
\renewcommand{\thetable}{\theenumi}

Question:If X follows a binomial distribution with parameters n = 5, p and
$p_X(2) = 9p_X(3)$
then p is?\\
\solution
\fi
\begin{align}
\mu&=np\\
&=5p
\end{align}
\begin{align}
\sigma^2&=np(1-p)\\
&=5p(1-p)
\end{align}
\begin{align}
	Y &\sim N\brak{\mu,\sigma}
\end{align}
Using the condition \(p_Y(2) = 9p_Y(3)\), we get:
\begin{align}
e^{-\frac{1}{2}(\frac{2-\mu}{\sigma})^2} &= 9 e^{-\frac{1}{2}\left(\frac{3-\mu}{\sigma}\right)^2} \\
\implies e^{-\frac{1}{2}\left(\frac{2 -  5p}{\sqrt{5p(1-p)}}\right)^2} &= 9 e^{-\frac{1}{2}\left(\frac{3 - 5p}{\sqrt{5p(1-p)}}\right)^2} \\
\implies e^{-\frac{1}{2}\left(\frac{2 - 5p}{\sqrt{5p(1-p)}}\right)^2} &= 9 e^{-\frac{1}{2}\left(\frac{3 - 5p}{\sqrt{5p(1-p)}}\right)^2} \\
\implies e^{-\frac{1}{2}\left(\frac{(2 - 5p)^2-(3 - 5p)^2}{(\sqrt{5p(1-p)})^2}\right)}&=9
\end{align}
Taking the natural logarithm of both sides, we have:
\begin{align}
\implies -\frac{1}{2}\left(\frac{(2 - 5p)^2-(3 - 5p)^2}{5p(1-p)}\right) &= \ln(9) \\
\implies 4+25p^2-20p-9-25p^2+30p &= -10p(1-p) \ln(9) \\
\implies 10p-5 &= -10p(1-p) \ln(9) \\
\implies 1-2p &= (2p-2p^2) \ln(9) \\
\implies 2p^2\ln(9) - 2p\ln(9) - 2p + 1 &= 0 
\end{align}
\begin{align}
p &= \frac{2\ln(9) + 2 \pm \sqrt{(-2\ln(9) - 2)^2 - 4(2\ln(9))(1)}}{2(2\ln(9))} \\
 &= \frac{2\ln(9) + 2 \pm \sqrt{4(\ln(9))^2 +4}}{4\ln(9)}\\
&=0.178211588
\end{align}
